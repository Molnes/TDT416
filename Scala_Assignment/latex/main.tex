\documentclass[11pt,a4paper]{report}
\usepackage[utf8]{inputenc}
\usepackage[T1]{fontenc}
\usepackage{alltt}
\usepackage{xcolor}
\usepackage{graphicx}
\usepackage{hyperref}
\usepackage{listings}
\usepackage{amsmath}

\definecolor{dkgreen}{rgb}{0,0.6,0}
\definecolor{gray}{rgb}{0.5,0.5,0.5}
\definecolor{mauve}{rgb}{0.58,0,0.82}

\lstset{
    frame=tb,
    language=scala,
    aboveskip=3mm,
    belowskip=3mm,
    showstringspaces=false,
    columns=flexible,
    basicstyle={\small\ttfamily},
    numbers=none,
    numberstyle=\tiny\color{gray},
    keywordstyle=\color{blue},
    commentstyle=\color{dkgreen},
    stringstyle=\color{mauve},
    frame=single,
    breaklines=true,
    breakatwhitespace=true,
    tabsize=3
    }

% Changed the section command to say "Task #"
\def\thesection{Task \arabic{section}}
% Added a familiar HTML command for new paragraph
\newcommand{\p}{\medskip\noindent}

% Meta, should be edited and filled in with relative information
\title{TDT4165 Programming Languages Scala Project Part 1}
\author{Jakob Holkestad Molnes}
\date{\today}

\begin{document}
\maketitle


% Use \section for tasks
\section{}

\begin{lstlisting}[language=scala]
def run() = {
    // a)
    var arr = Array[Int]()
    for (i <- 1 to 50) {
        arr = arr :+ i
    }
    println("Array: " + arr.mkString(", "))

    // b)
    println("Sum of the array: " + sumOfArray(arr))

    // c)
    println("Sum of the array using recursion: " + sumRecursive(arr))

    // d)
    println("Fibonacci of 10: " + Fibonacci(10))
    // The difference between Int and BigInt is that Int is a 32-bit signed integer, while BigInt can grow as large as you need it to be, depending on the size of the number you give it.
    }

    /**
    * Computes the sum of all elements in an array of integers.
    *
    * @param arr An array of integers whose elements are to be summed.
    * @return The sum of all elements in the input array.
    */
    def sumOfArray(arr: Array[Int]): Int = {
    var sum = 0
    for (i <- 0 until arr.length) {
        sum += arr(i)
    }
    return sum
    }

    /**
    * Computes the sum of an array of integers using recursion.
    *
    * @param arr The array of integers to sum.
    * @return The sum of the integers in the array. If the array is empty, returns 0.
    */
    def sumRecursive(arr: Array[Int]): Int = {
    if (arr.isEmpty) 0
    else arr.head + sumRecursive(arr.tail)
    }

    /**
    * Computes the Fibonacci number for a given position.
    *
    * @param n The position in the Fibonacci sequence.
    * @return The Fibonacci number at the given position.
    */
    def Fibonacci(n: BigInt): BigInt = {
    if (n <= 1) n
    else Fibonacci(n - 1) + Fibonacci(n - 2)
    }
\end{lstlisting}
d)
The difference between Int and BigInt is that Int is a 32-bit signed integer, while BigInt can grow as large as you need it to be, depending on the size of the number you give it.

\section{}

\begin{lstlisting}[language=scala]
def run() = {

// a)  assignment 3 task 1
var x1 = ArrayBuffer[Double]()
var x2 = ArrayBuffer[Double]()
var realSol = ArrayBuffer[Boolean]()

QuadraticEquation(2, 1, -1, realSol, x1, x2)
println("Real solutions: " + realSol)
println("x1: " + x1)
println("x2: " + x2)

x1.clear()
x2.clear()
realSol.clear()

QuadraticEquation(2, 1, 2, realSol, x1, x2)
println("Real solutions: " + realSol)
println("x1: " + x1)
println("x2: " + x2)

// assignment 3 task 4
val f = Quadratic(3, 2, 1)
println(f(2))
}

/**
* Solves the quadratic equation of the form ax^2 + bx + c = 0.
*
* @param a The coefficient of x^2.
* @param b The coefficient of x.
* @param c The constant term.
* @param realSol An ArrayBuffer to store whether the solutions are real (true) or not (false).
* @param x1 An ArrayBuffer to store the first solution (if it exists).
* @param x2 An ArrayBuffer to store the second solution (if it exists).
*
* The function calculates the discriminant (d) of the quadratic equation:
* - If d > 0, there are two distinct real solutions, which are added to x1 and x2.
* - If d == 0, there is one real solution, which is added to x1.
* - If d < 0, there are no real solutions, and realSol is updated to false.
*/
def QuadraticEquation(
    a: Double,
    b: Double,
    c: Double,
    realSol: ArrayBuffer[Boolean],
    x1: ArrayBuffer[Double],
    x2: ArrayBuffer[Double]
) = {
var d = b * b - 4 * a * c
if (d > 0) {
    realSol += true
    x1 += (-b + Math.sqrt(d)) / (2 * a)
    x2 += (-b - Math.sqrt(d)) / (2 * a)
} else if (d == 0) {
    realSol += true
    x1 += -b / (2 * a)
} else {
    realSol += false
}
}

/**
* Defines a quadratic function based on the coefficients a, b, and c.
*
* @param a The coefficient of the x^2 term.
* @param b The coefficient of the x term.
* @param c The constant term.
* @return A function that takes an integer x and computes the value of the quadratic equation a*x^2 + b*x + c.
*/
def Quadratic(a: Int, b: Int, c: Int): Int => Int = { (x: Int) =>
(a * x * x) + (b * x) + c
}
\end{lstlisting}
b)  Which differences did you find, if any?

A big difference between doing it in scala and oz, is that the default types in scala uses "val" in function definitions.
While in oz, you can use ?varname to define a variable that can be changed later.
You can use an ArrayBuffer to get around this in scala. An array buffer is a mutable array that can be changed later.

\section{}
\subsection{a)}
\begin{lstlisting}[language=scala]
    /**
    * Creates a new thread that will execute the given task.
    *
    * @param task A function with no arguments and no return value that represents the task to be executed.
    * @return A new Thread instance that will run the provided task when started.
    */
   def runner(task: () => Unit): Thread = {
     new Thread(() => task())
   }
\end{lstlisting}
\subsection{b)}
\subsubsection{What is this code supposed to do?}
It looks like this code is meant to increase value1 by 1 and deacrese value2 by 1, and then print the values of both variables,
equaling a sum of 1000. And it is supposed to do this with 2 threads.

\subsubsection{Is it working as expected?}
No, it is not working as expected. The output is not always 1000.

\subsubsection{What do you think it is happening?}
The problem is that the two threads are not synchronized, so they are not waiting for each other to finish before they print the values of the variables.
So thread 2 can print while thread 1 is in the middle of increasing value1. This can lead to the sum not being 1000. Since value2 havent been decreased yet. 

\subsubsection{Would it be possible to experience the same behavior in Oz? Why or why not?}
This could also happen in Oz if you dont use locks or semaphores to synchronize the threads.
\subsubsection{Give one of how this behavior could impact a real application}

In a real-world application, such race conditions could lead to inconsistent financial transactions. For instance, if this were a banking application that transfers money between accounts (similar to value1 and value2), concurrent access without proper synchronization could lead to incorrect balances being recorded. This could result in losing money or double withdrawals.

\subsection{c)}
\begin{lstlisting}[language=scala]
object ConcurrencyTroubles {
    private var value1: Int = 1000
    private var value2: Int = 0
    private var sum: Int = 0
    
    def moveOneUnit(): Unit =  synchronized {
        value1 -= 1
        value2 += 1
        if(value1 == 0) {
            value1 = 1000
            value2 = 0
        }
    }
    
    def updateSum(): Unit = {
        sum = value1 + value2
    }
    
    def execute(): Unit = {
        while(true) {
        moveOneUnit()
        updateSum()
        Thread.sleep(50)
        }
    }
    
    def main(args: Array[String]): Unit = {
        for (i <- 1 to 48) {
        val thread = new Thread {
            override def run = execute()
        }
        thread.start()
        }
        
        while(true) {
        updateSum()
        println(s"$sum [$value1 $value2]")
        Thread.sleep(100)
        }
    }
    }
\end{lstlisting}
You can simply fix it by adding the synchronized keyword to the moveOneUnit function. This will make sure that only one thread can access the function at a time. This will make sure that the values are updated correctly.

\begin{lstlisting}
import java.util.concurrent.atomic.{AtomicInteger, AtomicReference}

object ConcurrencyTroubles {
    private val value1: AtomicInteger = new AtomicInteger(1000)
    private val value2: AtomicInteger = new AtomicInteger(0)
    private val sum: AtomicReference[Int] = new AtomicReference(0)

    def moveOneUnit(): Unit = {
    value1.decrementAndGet()
    value2.incrementAndGet()
    if(value1.get() == 0) {
        value1.set(1000)
        value2.set(0)
    }
    }

    def updateSum(): Unit = {
    sum.set(value1.get() + value2.get())
    }
    
    def execute(): Unit = {
    while(true) {
        moveOneUnit()
        updateSum()
        Thread.sleep(50)
    }
    }

    def main(args: Array[String]): Unit = {
    for (i <- 1 to 48) {
        val thread = new Thread {
        override def run = execute()
        }
        thread.start()
    }
    
    while(true) {
        updateSum()
        println(sum.get() + " [" + value1.get() + " " + value2.get() + "]")
        Thread.sleep(100)
    }
    }
}
     
\end{lstlisting}
You can also use AtomicIntegers to make sure that the values are updated correctly.

\end{document}