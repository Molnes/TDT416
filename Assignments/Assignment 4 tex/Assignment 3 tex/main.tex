\documentclass[11pt,a4paper]{report}
\usepackage[utf8]{inputenc}
\usepackage[T1]{fontenc}
\usepackage{alltt}
\usepackage{xcolor}
\usepackage{graphicx}
\usepackage{hyperref}
\usepackage{listings}
\usepackage{amsmath}

\lstset{
    backgroundcolor=\color{green!10},
    basicstyle=\ttfamily,
    breaklines=true,
    columns=fullflexible,
    frame=single,
    keepspaces=true,
    numbers=left,
    numbersep=5pt,
    numberstyle=\tiny\color{gray},
    showstringspaces=false,
    showtabs=false,
    tabsize=4
    }

% Changed the section command to say "Task #"
\def\thesection{Task \arabic{section}}
% Added a familiar HTML command for new paragraph
\newcommand{\p}{\medskip\noindent}

% Meta, should be edited and filled in with relative information
\title{TDT4165 Programming Languages Assignment 4}
\author{Jakob Holkestad Molnes}
\date{\today}

\begin{document}
\maketitle

% Use \section for tasks
\section{}
\subsection{a)}
30
10
20
3000
100
200
\subsection{b)}
30 (from {System.show C})
10 (from the first thread)
20 (from the second thread)
3000 (from {System.show C * 100})
100 (from the first thread after 100 ms delay)
200 (from the second thread after 100 ms delay)

It is possible that the output is not in the order of the threads, as the threads are running concurrently. This means that the threads are not guaranteed to finish in the order they were started. 

\subsection{c)}
2
20
22
\subsection{d)}
2 (from the first thread where A is assigned to 2)
20 (from the second thread where B is assigned to A * 10)
22 (from the main thread where C is assigned to A + B)

It cant print in any other order, since C relies on A and B, and B relies on A.
Oz waits on unbound variables, so it will wait for A to be assigned before it can assign B, and it will wait for B to be assigned before it can assign C.

\section{}
\subsection{a)}
\begin{lstlisting}[language=Oz]
fun {Enumerate Start End} UTail in
    thread 
        if Start =< End then
            thread UTail = {Enumerate Start+1 End} end
            Start|UTail
        else 
            nil
        end
    end
end
\end{lstlisting}

\subsection{b)}
\begin{lstlisting}[language=Oz]
fun {GenerateOdd Start End} IntList in 
    IntList = {Enumerate Start End}
    thread {List.foldR IntList fun {$ X Y} if {Int.isOdd X} 
        then X|Y 
        else Y end 
    end nil }
    end
end
\end{lstlisting}

\subsection{c)}
Both functions return a \_<optimized> when printed through show. The reason for this is that Oz's compiler detects the use of asynchronous recursion and lazy evaluation due to the thread keyword within Enumerate. By generating the list asynchronously and evaluating elements only as needed, the compiler applies an optimization, showing it with the <optimized> output.

\section{}
\subsection{a)}
\begin{lstlisting}[language=Oz]
fun { ListDivisorsOf Number} NumberList in
    NumberList = { Enumerate 1 Number}
    thread { List.foldR NumberList fun {$ X Y} if { Int.'mod' Number X} == 0 then X | Y else Y end end
    nil } end
end
\end{lstlisting}


\subsection{b)}

\begin{lstlisting}[language=Oz]
fun {ListPrimesUntil N} NumberList in
    NumberList = { Enumerate 1 N}
    thread { List.foldR NumberList fun {$ X Y} thread if
    { List.length { ListDivisorsOf X}} == 2 then X |
    Y else Y end end end nil } end
end
\end{lstlisting}

\section{}
\subsection{a)}
\begin{lstlisting}[language=Oz]
fun {EnumerateLazy} Func in
    fun lazy {Func N}
        N | {Func N+1}
    end
   1 | {Func 2}
end
\end{lstlisting}

\subsection{b)}
\begin{lstlisting}[language=Oz]
fun {Primes} Func StartNumberList in
    fun lazy {Func N NumberList}
        case NumberList of H | T then
            case {ListDivisorsOf N} of _|_|_|_ then
                {Func H T}
            else
                N | {Func H T}
            end
        end
    end
    StartNumberList = {EnumerateLazy}
    case StartNumberList of _|X2|X3|T then
        X2 | {Func X3 T}
    end
end
\end{lstlisting}


   \end{document}