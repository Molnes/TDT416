\documentclass[11pt,a4paper]{report}
\usepackage[utf8]{inputenc}
\usepackage[T1]{fontenc}
\usepackage{alltt}
\usepackage{graphicx}
\usepackage{hyperref}
\usepackage{listings}
\usepackage{amsmath}

% Changed the section command to say "Task #"
\def\thesection{Task \arabic{section}}
% Added a familiar HTML command for new paragraph
\newcommand{\p}{\medskip\noindent}

% Meta, should be edited and filled in with relative information
\title{TDT4165 Programming Languages Assignment 2}
\author{Jakob Holkestad Molnes}
\date{\today}

\begin{document}
\maketitle

% Use \section for tasks
\section{}
My mdc works by first Lexing the postfix string into a list of lexemes. The lexemes are then tokenized into a list of tokens, where each token is either a number, an operator, or a command. The tokens are then interpreted by pushing numbers onto a stack, and when an operator is encountered, the two top elements of the stack are popped and used as the operands for the operator. The result of the operation is then pushed back onto the stack. The special characters "p", "d", "i", and "c" are used to print the stack, duplicate the top element of the stack, invert the top element of the stack, and clear the stack, respectively. This process is repeated until all tokens have been interpreted, and the final result is the top element of the stack.


\begin{lstlisting}[language=Oz]
   % Task 1
   % a)
   fun {Lex Input}
      {String.tokens Input 32}
   end
   
   {System.showInfo "Task 1a: "}
   {Show {Lex "1 2 + 3 *"}}
   
   
   % b)
   
   fun {Tokenize Lexemes}
      case Lexemes of Head|Tail then 
           if {String.isFloat Head} then
              number(1:{String.toFloat Head}) 
           elseif {List.member Head ["+" "-" "/" "*"] } then 
              operator(type: case Head of "+" then "plus" 
                          [] "-" then "minus" 
                          [] "*" then "multiply" 
                          [] "/" then "divide" end) 
           elseif {List.member Head ["p" "c" "d" "i"]} then
              command(type: case Head of "p" then "print" 
                      [] "i" then "inverse" 
                      [] "c" then "clear" 
                      [] "d" then "duplicate" end)
           end
           | {Tokenize Tail}
      else 
          nil
      end
   end
   
   {System.showInfo "Task 1b: "}
   {Show {Tokenize {Lex "1 2 + 3 *"}}}
   
   
   % c)
   fun {Interpret Tokens}
      {TrueInterpret Tokens nil}
   end
   
   fun {TrueInterpret Tokens Stack}
      case Tokens of operator(type:C)|Rest then
          case Stack of N1|N2|Tail then
              if C == "plus" then
                  {TrueInterpret Rest (N1+N2 | Tail)}
              elseif C == "minus" then
                  {TrueInterpret Rest (N1-N2 | Tail)}
              elseif C == "divide" then
                  {TrueInterpret Rest (N1/N2 | Tail)}
              elseif C == "multiply" then
                  {TrueInterpret Rest (N1*N2 | Tail)}
              end
          end
      [] number(N)|Tail then
          {TrueInterpret Tail (N | Stack)}
      [] command(type:Action)|Tail then
          if Action == "print" then
              {Show Stack}
              {TrueInterpret Tail Stack}
          elseif Action == "duplicate" then
              {TrueInterpret Tail (Stack.1 | Stack)}
          elseif Action == "inverse" then
              case Stack of Head|Rest then
                  {TrueInterpret Tail (~Head | Rest)}
              end
          elseif Action == "clear" then
              {TrueInterpret Tail nil}
          end
      else
          Stack
      end
   end
   
   
   {System.showInfo "Task 1c: "}
   {Show {Interpret {Tokenize {Lex "1 2 3 +"}}}}
   
   
   
   % d)
   
   {System.showInfo "Task 1d: "}
   {Show {Interpret {Tokenize {Lex "1 2 p 3 +"}}}}
   
   
   % e)
   
   {System.showInfo "Task 1e: "}
   {Show {Interpret {Tokenize {Lex "1 2 3 + d"}}}}
   
   % f)
   
   {System.showInfo "Task 1f: "}
   {Show {Interpret {Tokenize {Lex "1 2 3 + i"}}}}
   
   % g)
   
   {System.showInfo "Task 1g: "}
   {Show {Interpret {Tokenize {Lex "1 2 3 + c"}}}}
   
\end{lstlisting}


\section{}
\begin{lstlisting}
    % Task 2
% a)
fun {ExpressionTreeInternal Tokens ExpressionStack} NewExpr in
   case Tokens of operator(type:Op)|Rest then
      case ExpressionStack of E1|E2|Tail then
         NewExpr = case Op of
                      "plus" then plus(E1 E2)
                   [] "minus" then minus(E1 E2)
                   [] "multiply" then multiply(E1 E2)
                   [] "divide" then divide(E1 E2)
                   end
         {ExpressionTreeInternal Rest (NewExpr | Tail)}
      end
   [] number(N)|Rest then
      {ExpressionTreeInternal Rest (N | ExpressionStack)}
   else
      ExpressionStack.1
   end
end

{System.showInfo "Task 2a: "}
{Show {ExpressionTreeInternal {Tokenize {Lex "2 3 + 5 /"}} nil}}

% b)

fun {ExpressionTree Tokens}
   {ExpressionTreeInternal Tokens nil}
end

{System.showInfo "Task 2b: "}
{Show {ExpressionTree {Tokenize {Lex "3 10 9 * - 7 +"}}}}
\end{lstlisting}

To explain how the conversion works in a high-level description, we can say that the conversion from postfix to a tree structure is done by first tokenizing the input string, then interpreting the tokens into a tree structure. The tree structure is built by recursively interpreting the tokens, and building the tree from the bottom up. The tree is built by pushing numbers onto a stack, and when an operator is encountered, the two top elements of the stack are popped and used as the operands for the operator. The result of the operation is then pushed back onto the stack. This process is repeated until all tokens have been interpreted, and the final result is the root of the tree.


% Command for next page
\clearpage

\section{}

\subsection{a)}
\textit{Describe the grammar of the lexemes in Task 1 using a formal notation. This means not using EBNF,
but using the formal notation with sets and tuples. Refer to slide 53 in Lecture 2.}


 The grammar of the lexemes in Task 1 can be described as a set of tuples, where each tuple consists of a string and a type. The type can be either a number, an operator, or a string. The grammar can be described as follows:
\begin{lstlisting}[mathescape,escapechar=\%]
   V = {root}
   S = {1, 2, 3, 4, 5, 6, 7, 8, 9, 0, +, -, *, /, p, c, d, i}
   R = {root, S}
   \end{lstlisting}

   \subsection{b)}
   \textit{Describe the grammar of the records returned by the ExpressionTree function in Task 2, using (E)BNF}
   
   The grammar for the records returned by the ExpressionTree function in Task 2 can be described using the following EBNF:
   \begin{lstlisting}[mathescape,escapechar=\%]
   start       ::==    <number> 
               |       <operation>
   
   <operation> ::==    <operator>(<number> <number>) 
               |       <operator>(<operation> <operation>) 
               |       <operator>(<number> <operation>) 
               |       <operator>(<operation> <number>)
   
   <operator>  ::==    plus 
               |       minus 
               |       multiply 
               |       divide
   
   <number>    ::==    <number><number>
               | 1 | 2 | 3 | 4 | 5 | 6 | 7 | 8 | 9 | 0
   \end{lstlisting}
   
   \subsection{c)}
   \textit{
   Which kind of grammar is the grammar you defined in step a)? Is it regular, context-free, context-sensitive, or unconstrained? What about the one from step b)?
   }

   I think step A) is regular grammar due to the simplicity of tokens. While task B) is context-free due to the recursive nature of the expression tree.
   
   \end{document}